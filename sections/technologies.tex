\section{Angular}
Angular wurde als zentrales Frontend-Framework für die Umsetzung der Webanwendung eingesetzt. \cite{angular_docs} 
Ziel war die Entwicklung einer strukturierten, wartbaren und erweiterbaren Benutzeroberfläche, die eine zentrale Verwaltung von Mediatheken-Abonnements ermöglicht.
Durch den komponentenbasierten Aufbau eignet sich Angular besonders für komplexe Anwendungen mit klar getrennten Funktionsbereichen \cite{angular_architecture}.


\subsection{Rolle von Angular innerhalb der Anwendung}
Angular übernimmt die vollständige Darstellung und Interaktion der Benutzeroberfläche.
Sämtliche Funktionen zur Verwaltung von Abonnements, zur Anzeige von Laufzeiten sowie zur Darstellung der Verfügbarkeit von Filmen und Serien werden im Frontend realisiert.
Die Trennung von Frontend und Backend ermöglicht eine klare Verantwortungsaufteilung zwischen Präsentations- und Geschäftslogik.

\subsection{Architektur als Single-Page-Application}
Die Anwendung wurde als Single-Page-Application (SPA) umgesetzt, bei der Inhalte nach dem initialen Laden dynamisch aktualisiert werden, ohne dass ein vollständiger Seitenreload erforderlich ist \cite{mdn_spa}.
Dieses Konzept verbessert die Benutzererfahrung erheblich, da Navigationsvorgänge zwischen verschiedenen Bereichen verzögerungsfrei erfolgen.

Für eine Abo-Verwaltungsplattform ist dieses Verhalten besonders relevant, da häufig zwischen unterschiedlichen Ansichten gewechselt wird, um Laufzeiten zu vergleichen oder Inhalte zu prüfen. 

\subsection{Begründung der Wahl von Angular gegenüber React}
Die Entscheidung für Angular fiel aufgrund des ganzheitlichen Framework-Ansatzes.
Angular stellt im Gegensatz zu React ein vollständiges Framework bereit, das unter anderem Routing, Formularverarbeitung und HTTP-Kommunikation standartmäßig integriert \cite{freeman_angular}.
Dies ermöglicht eine einheitliche Projektstruktur und erleichtert die langfristige Wartung der Anwendung. 

\subsection{Komponentenstruktur der Abo-Verwaltung}
Die Anwendung ist in logisch getrennte Komponenten unterteilt. Zentrale Komponenten sind unter anderem:

\begin{itemize}
    \item Übersicht der aktiven Abonnements
    \item Detailansicht einzelner Abos mit Laufzeitinformationen
    \item Ansicht zur Anzeige von Filmen und deren Streaming-Anbietern
    \item Benutzer- und Kontoverwaltung
\end{itemize}

Diese Struktur unterstützt eine klare Trennung der Verantwortlichkeiten und erleichtert zukünftige Erweiterungen der Anwendung.

\subsection{Data Binding und Services zur Zustandsverwaltung}
Angular stellt mit Data Binding und Services bewährte Mechanismen zur Verwaltung des Anwendungszustands bereit \cite{angular_architecture}.
Benutzereingaben werden direkt mit dem Datenmodell verknüpft, wodurch Änderungen unmittelbar in der Benutzeroberfläche sichtbar werden.

Die Kommunikation mit dem Backend erfolgt über dedizierte Services, die sämtliche HTTP-Anfragen kapseln und eine konsistente Datenbasis sicherstellen.


\section{Quarkus}
\subsection{Grundlagen des Frameworks}
\subsection{Vorteile im Kontext der Abo-Verwaltung}
\subsection{Aufbau der REST-Schnittstellen}
\subsection{Kommunikation mit externen APIs}

\section{Keycloak}
\subsection{Grundlagen der Authentifizierung und Autorisierung}
\subsection{Ablauf des Login- und Token-Prozesses}
\subsection{Rollen- und Benutzerverwaltung}
\subsection{Sicherheitsaspekte und Datenschutz}

\section{PostgreSQL}
\subsection{Datenmodell für Abonnements und Mediatheken}
\subsection{Strukturierung der relationalen Tabellen}
\subsection{Vorteile relationaler Datenbanken für dieses Projekt}
\subsection{Evaluierung möglicher Alternativen}

\section{Docker}
\subsection{Grundlagen der Containerisierung}
\subsection{Einsatz von Docker im Entwicklungsprozess}
\subsection{Vereinheitlichung der Entwicklungsumgebung}

\section{Tailwind CSS}
\subsection{Utility-First-Ansatz}
\subsection{Gestaltung der Benutzeroberfläche}
\subsection{Integration von Tailwind CSS in Angular}
\subsection{Beitrag zur Benutzerfreundlichkeit}

\section{GitHub}
\subsection{Versionsverwaltung im Projekt}
\subsection{Branching-Strategie und Zusammenarbeit}
\subsection{Nachvollziehbarkeit der Entwicklung}
